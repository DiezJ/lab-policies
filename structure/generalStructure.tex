\documentclass[12pt]{article}
\usepackage[top=0.8in, bottom=0.8in, right=0.8in, left=0.8in,
paperwidth=8.5in, paperheight=11in, nohead]{geometry}
\geometry{letterpaper}
\usepackage[pdftex]{graphicx}
\usepackage{color}
\usepackage[normalem]{ulem}
\usepackage{amssymb}
\usepackage{amsmath}
\usepackage{epstopdf}
\usepackage{setspace}
\usepackage{mdwlist}
\usepackage{hyperref}

\title{Lab structure}
\author{Lauren C. Ponisio}

\begin{document}
\maketitle

\section{Lab meetings}
\begin{enumerate}
\item weekly lab meetings (bi-weekly in year 1, alternating with
  one-on-one meetings)
\item topics ranging from professional development, academic
  survivorship, current projects, relevant literature, reproducible
  science and analytic tools
\item we being each lab meeting with a check in: ``how are you doing
  academically? How are you doing as a person?''
\item thematic groups based on ongoing projects will also meet every
  other week
\end{enumerate}

\section{Meetings with Lauren}
\begin{enumerate}
\item each student/post-doc meets with Lauren for two hours a month,
  which can be either in one chunk (to go over code) or two, one hour
  meetings
\item possible meeting uses:
  \begin{enumerate}
  \item side-by-side code together, working through analyses or bugs
    in models
  \item refine a conceptual framework
  \item discuss professional development goals
  \item outline the framing of an article
  \item discussion funding options
  \end{enumerate}
\end{enumerate}

\section{Submitting grants}
\begin{enumerate}
\item as per UC policy, Lauren is a PI on all grants originating from
  the lab (excluding fellowships and ``award'' like grants such as the
  NGS young explorer grant)
\item all grants must be approved by the sponsored programs
  office. They need 3 business days before the grant is due to
  approve the grant.
\item Our grant officer is Laura A Schulte
  <laura.schulte@ucr.edu>. Address all questions regarding budgets
  (i.e., the graduate student step system for stipends) to him/her
\end{enumerate}

\section{Requesting comments}
\begin{enumerate}
\item For manuscripts, be sure to storyboard the manuscript with
  Lauren before any serious writing.
\item Have a peer (i.e., post-doc or graduate student in the lab or
  with similar research interests) read the draft before submitting it
  to Lauren.
\item Add your article to the
  \href{https://drive.google.com/a/ucr.edu/file/d/0B32TI_00hnPfVDJmVk5pR045Tnc/view?usp=sharing}{manuscript
    queue}. Expect review to take two weeks or more depending on how
  many manuscripts are in the queue.
\item Lab members are required to write manuscripts in Latex.
\end{enumerate}

\section{Submitting manuscripts}
\begin{enumerate}
\item All coauthors must read and sign off on a manuscript before
  submission. Online submission is done by the first author.
\item before submission, re run all code, check reported statistics in
  manuscript
\item posting code and data (a subset if necessary) is a lab
  requirement
\item check references, and make sure journal names are abbreviated
\item when responding to reviews, send out revisions and latex diff
  file to all authors and give them a time line to comment, and only
  after all have responded (either with comments or by confirming they
  don't have comments) re-submit.
\end{enumerate}

\section{Signatures}
There is an inbox outside of Lauren's office. Put had copies that need
signatures inside with a tab marking the line the signature is
needed. I will sign it and place it in the outbox.

\section{Requesting letters of recommendation}
Add the letter you need to the lab letter drive spreadsheet. Include
the date it is due, and detailed instructions on how it needs to be
submitted (the email, the link, etc.) to the spreadsheet. Letters must
be requested \textbf{one month in advance}.
\end{document}

%%% Local Variables:
%%% mode: latex
%%% TeX-PDF-mode: t
%%% End:


