\documentclass[12pt]{article}
\usepackage[top=0.8in, bottom=0.8in, right=0.8in, left=0.8in,
paperwidth=8.5in, paperheight=11in, nohead]{geometry}
\geometry{letterpaper}
\usepackage[pdftex]{graphicx}
\usepackage{color}
\usepackage[normalem]{ulem}
\usepackage{amssymb}
\usepackage{amsmath}
\usepackage{epstopdf}
\usepackage{setspace}
\usepackage{mdwlist}
\usepackage{hyperref}
\usepackage{xr}

\title{Expectations}
\author{Lauren C. Ponisio}

\begin{document}

\maketitle

\section{Authorship}
There are five key parts of a study:
\begin{enumerate}
\item Idea generation
\item Funding
\item Data collection (including cleaning and curation)
\item Analysis
\item Writing
\end{enumerate}

A good rule of thumb is a coauthor must be involved in at least three
of the five pieces. It is widely acknowledged that underrepresented
groups in science are coauthored less because their contributions are
not as easily valued or apparent. Lab members are encouraged to keep
this in mind when considering the contributions of others, and also
when advocating for their own contributions.

We are part of community and should freely help each other whenever
appropriate, i.e., bounce ideas off each other, ask for help with
code, ran a statistical analysis by someone. That being said, it is
important to discuss coauthorship early and often. Be upfront about:

\begin{enumerate}
\item Your interest in a project
\item What you can contribute ``freely''v (i.e., without coauthorship)
\item What level of involvement you believe warrants coauthorship
  (i.e., I am happy to brain storm an analysis with you, but if I am
  developing code with you I would like to be more fully involved in
  the project and be coauthored)
\end{enumerate}

\section{Reproducibility}
We are committed to using the best practices in scientific computing
and reproducible science. All the
materials needed to reproduce the study entirely (from data collection
to analysis), must be make available publicly and associated with all
publications. This includes:
\begin{enumerate}
\item 1. Curating all data in a relational database where automated
  data cleaning is conducted that is version controlled, including a
  record of any changes to the data and the raw data is never altered.
\item Version controlling and posting all protocols (collection,
  molecular, etc.) publicly, and associate each protocol DOI with its
  corresponding publication.
\item Version control all analytic code. Final products will be
  executable from a single script with an explanatory Rmarkdown file,
  publicly available and associated with corresponding publication.
\item Post all final data publicly
\end{enumerate}

\section{Data management}
All lab members are responsible for the curation of the data they
collect. The lab's data management protocol is:

\begin{enumerate}
\item Data will first be entered from field sheets into a spreadsheet
  on UC Drive, the UC’s version of Google Drive with FERPA and US-EU
  Safe Harbor protections, then downloaded as \.csv files into
  password-protected Dropbox (Box is also an option, but access
  disappears after your UC id expires).
\item We will use R (or programming language of your choice) to check
  each datasets fields for consistency and typographic errors and
  output the cleaned data files for analyses. After any updates, the
  Google spreadsheet is downloaded as a \.csv file, and cleaned and
  assembled using R protocols, while never transforming the original
  Google spreadsheet. All code for cleaning data will be version
  controlled on Github so there is a record of any changes made to the
  data.
\item Structure Query Language (SQL) Lite package in R will create a
  relational database that unites each of the relevant datasets for
  analysis. Traditional spreadsheets will be created for the data made
  publicly available.
\end{enumerate}

\section{Science communication}
All lab members are required to communicate their research to a non
academic audience at least once a year (i.e., blog for National
Geographic, presentation at the Entomology fair, presentation to a
school group etc.).




\end{document}

%%% Local Variables:
%%% mode: latex
%%% TeX-PDF-mode: t
%%% End:


