\documentclass[12pt]{article}
\usepackage[top=0.8in, bottom=0.8in, right=0.8in, left=0.8in,
paperwidth=8.5in, paperheight=11in, nohead]{geometry}
\geometry{letterpaper}
\usepackage[pdftex]{graphicx}
\usepackage{color}
\usepackage[normalem]{ulem}
\usepackage{amssymb}
\usepackage{amsmath}
\usepackage{epstopdf}
\usepackage{setspace}
\usepackage{mdwlist}
\usepackage{hyperref}
\usepackage{xr}

\title{General Lab Expectations}
\author{Diez Lab}

\begin{document}

\maketitle

This document is a modified version of text developed by Lauren Ponisio and Lauren Hallett.

\section{Authorship}
There are five key parts of a study:
\begin{enumerate}
\item Idea generation
\item Funding 
\item Data collection (including cleaning and curation)
\item Analysis
\item Writing
\end{enumerate}

A good rule of thumb is a coauthor is involved in at least three of the five pieces, or has contributed substantially to two. When an individual's support or input is sought for an early or critical stage of a project, it is lab practice to give them the opportunity to be involved to the level of authorship. For example, if the project involves follow-up at someone's study site or relies on their expertise for system knowledge/experimental design/methodological approach, the project lead should make sure that they have the opportunity to continue to engage with the project should they be interested. 

It is widely acknowledged that underrepresented groups in science are coauthored less because their contributions are
not as easily valued or apparent. Lab members are encouraged to keep this in mind when considering the contributions of others, and also when advocating for their own contributions. Especially when undergraduates are involved in a project, they should be offered the chance to remain involved past graduation (if they are motivated and interested to do so). 

Lead authors are expected to keep a running tally of folks who contribute to the project to ensure that everyone is appropriately credited (either as authors or in the acknowledgement). It is recommended that this tally be maintained and updated in a project wiki on Github.


\section{Meetings with Jeff}
\begin{enumerate}
\item I'm happy to have a regular, individual meeting with everyone in the lab as often as once each week. Typically these will be longer and more consistent early in a project, shorter and/or held on an as-need basis during active data collection, and more regular again during analysis, writing, and follow-up.
\item It is highly encouraged that you develop an agenda for the meeting and outline your goals for the meeting at the top (Jeff may add agenda items at this time too). 
\item First year graduate students will initially use this time as an informal one-on-one reading group with Jeff to explore the literature, hone your interests, and work toward potential thesis questions. Subsequently the meetings can be used to discuss experimental design, analysis, proposals and drafts, etc. 
\item Periodically these meetings should be used to discuss longer-term goals and professional development (generally it's best to hold these types of meetings when you first join the lab and after each six-month plan exercise) to make sure that everyone's goals are being met.
  \end{enumerate}

\section{Lab retreats - let's discuss this in fall 2021}
\begin{enumerate}
\item Lab retreats are held at least annually.
\item Lab retreats are scheduled for times when many members have either proposals or manuscript drafts ready for larger group review (often July-August).
\item "Writing workshops" are a core component of lab retreat, with each person submitting either a draft or extended figures, and the group workshopping each in turn. See the "write workshop" guidelines for lab protocol on the structure of our in-person review process (we use this process both in lab meetings and retreats).
\item Other types of work activities in lab retreats can included outlining or working on lab papers, checking in on core aspects of lab functioning (e.g., data management, lab inventory), and reflecting on and discussing short and long-term goals. 
\item Lab retreats end with something fun! Ideally retreats are an actual retreat (we go somewhere interesting), but at a minimum should involve a lab bonding activity at the end.
\end{enumerate}



\section{Six month plans & IDPs}
\begin{enumerate}
\item Six month plans are an opportunity to reflect on your short and long term goals (really they include six-month plans as well as one-year and five-year goals). You can use whatever structure works best for you - some people like to list items to accomplish and the anticipated time investment of each, others like to make calendars with sequential goals. 
\item Six month plans are discussed as a full group. It is really helpful for everyone to know what everyone else is up to for maximum collaboration and support. It is also helpful to get feedback on the realism of different plans (e.g., how long will it take you to sort biomass? Someone else in the lab will be able to advise). 
\item Six month plans are also a way for both you and Jeff to make sure you are progressing in a timely way. Unforseen opportunities and obstacles may arise over six months, but as a general rule assessing your progress against your six month plan is a good way to take stock of what is and is not working, and to identify paths to address it.
\end{enumerate}


\section{Submitting grants}
\begin{enumerate}
\item Typically, per university policy, Jeff is a PI on all grants originating from
  the lab (excluding fellowships and ``award'' style grants aimed at grads/postdocs. Postdocs can be PIs with some funding agencies, however; if this applies Jeff will work with you to make sure you can get official credit as a PI or co-PI.
\item All grants must be approved by the sponsored programs office. They need \textit{at least} 3 business days before the grant is due to approve the grant. It is recommended that you get materials uploaded well in advance. It is most important to have the budget and budget recommendation uploaded with a good time buffer; the project narrative can be in draft form until closer to the final deadline.  
\item We have an assigned grant officer.  Address all questions regarding budgets (i.e., the graduate student step system for stipends) to him/her. Start working on budget one month prior to grant submission.
\item Please be courteous with the IEE and sponsored projects grants staff! They have to deal with stressed out people a lot, let's make their lives easier by being polite in our requests for help and timely with our grant item uploads.
\end{enumerate}

\section{Requesting comments}
\begin{enumerate}
\item For manuscripts, be sure to storyboard the manuscript with Jeff before any serious writing.
\item You may want to have a peer (i.e., post-doc or graduate student in the lab or with similar research interests) read the draft before submitting it to Jeff.
\item \textbf Expect review to take up to two weeks depending on timing.
\end{enumerate}

\section{Submitting manuscripts}
\begin{enumerate}
\item All coauthors must read and sign off on a manuscript before
  submission. Online submission is done by the first author.
\item Before submission, re run all code and check the reported statistics in
  manuscript.
\item Before publication (and ideally before submission), check that your workflow is public and interpretable (e.g., well commented and with the order of analyses well explained. For complicated workflows, an explanation in a README is useful; it can also be helpful to break the code into files associated with each major analysis or figure.
\item Share communication from the journal with your coauthors in a timely way (e.g., reviews, acceptances or rejections).
\item When responding to reviews, include a cover letter that explains major changes, a line-item response to each comment, and either a tc or latex diff file of the manuscript. It can be very helpful to have a meeting with coauthors upon receiving a review to discuss a revision plan (rather than stewing with it alone and only asking for feedback after a first pass), and then sending the full revised package back out after completion. Only resubmit after all have responded (either with comments or by confirming they don't have comments). If an external coauthor is particularly difficult to reach, it can be helpful to give them a fair deadline (e.g., "if I do not hear from you in the next two weeks I will assume you do not have comments and are fine with submission").
\end{enumerate}


\section{Requesting letters of recommendation}
Please give Jeff a good lead time on letters of recommendation (ideally at least one month in advance). Let him know what it is for, when it is due, and detailed instructions on how it needs to be submitted (the email, the link, etc.). It is also very useful to follow up with your latest CV and some talking points you would like to have highlighted. \textbf{Please remind Jeff of deadlines/ask for submission confirmation as the deadline approaches.} Also, if you are requesting a batch of letters (e.g., for graduate school or faculty jobs), it is really helpful to provide a google spreadsheet of each application, its deadline, and its submission method so that one doesn't get missed.


\section{Reproducibility}
We are committed to using the best practices in scientific computing and reproducible science. All the materials needed to reproduce the study entirely (from data collection to analysis) should be made public upon publication. This includes:
\begin{enumerate}
\item Cleaned data products and associated metadata, uploaded to an appropriate digital archive. Often this will be the Environmental Data Initiative or the Knowledge Network for Biodiversity.
\item Well-documented analytical code to run the  paper's analyses, archived either on Github or Zenodo.
\end{enumerate}

\section{Data management}
to be developed...

\end{document}

%%% Local Variables:
%%% mode: latex
%%% TeX-PDF-mode: t
%%% End:


