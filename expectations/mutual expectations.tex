\documentclass[12pt]{article}
\usepackage[top=0.8in, bottom=0.8in, right=0.8in, left=0.8in,
paperwidth=8.5in, paperheight=11in, nohead]{geometry}
\geometry{letterpaper}
\usepackage[pdftex]{graphicx}
\usepackage{color}
\usepackage[normalem]{ulem}
\usepackage{amssymb}
\usepackage{amsmath}
\usepackage{epstopdf}
\usepackage{setspace}
\usepackage{mdwlist}
\usepackage{hyperref}
\usepackage{xr}


\title{Mutual lab expectations}
\subtitle{ }
\author{Diez Lab}

\begin{document}

\maketitle

This document is a modified version of text developed by Lauren Ponisio and Lauren Hallett.

\section{PI - What you can expect of Jeff}
My overall goals with our lab group are: (i) to conduct high quality research, and (ii) enable undergraduate and graduate students to succeed, broadly defined as having a productive time here at UO and being prepared and competitive for your next move.  With those in mind, here's what you can expect of me.

\begin{enumerate}
\item Foster and encourage an atmosphere of inclusivity, collaboration, exploration, learning, and good communication.
\item Check in on and forward the professional interests of the members of the lab, including expanding your professional networks.
\item Ensure that public dollars are well-spent; that the science that emerges from the lab is productive, rigorous, and cutting edge.

\item Provide mentorship on research projects, from experimental design through to publication, and mentorship on collaborations and career trajectories.
\item Be aware of your progress and provide structure to keep you on track, including feedback when things are going well and shining light on areas for growth.
\item Be timely in feedback on abstracts and manuscripts (typically within a 2 week period).
It's generally expected that Jeff should approve all abstracts, manuscripts, or other representation of the research that comes from the lab, before it leaves the door for submission to meetings or journals.
\item Be timely and communicative when writing letters of recommendation.
\item Ultimately be responsible for research funding, accounting,  scientific oversight, and training and oversight of personnel.
\end{enumerate}


\section{Graduate Students - What Jeff expects of you}
\begin{enumerate}
\item Be a leader in the lab and contribute to an inclusive and collegial environment. This includes:
\begin{enumerate}
\item Attending and engaging in lab functions (lab meetings, lab retreats, etc) 
\item Supporting one another in the field 
\item Being welcoming to new members and ``Paying it forward" in terms of teaching others protocols and methods
\item Providing feedback on each other's experimental design, proposals, talks, and drafts
\item Treating one another with courtesy and respect
\end{enumerate}

\item Set your own schedule and communicate and develop your goals with Jeff; this is best done through six month plans and an individual develop plan (IDP) if you choose to have one. We will use these plans to make sure we are on the same page about both your research and professional goals and to benchmark your progress.

\item Develop your own research ideas; this involves spending time with the literature, deep thinking and brainstorming, and being receptive and responsive to feedback from Jeff and the lab.

\item Request feedback on abstracts, proposals, and drafts in a timely way (ideally by providing the document with plenty of lead time; during crunch periods by communicating with Jeff about when he can anticipate the document). 

\item Take initiative in identifying additional mentors and collaborators as needed and discussing these options with Jeff, who will provide guidance in how to reach out and expand your mentor network.

\item Be flexible, available and willing to put in the time and work in emergencies or crunch times (particularly field season).

\item Make an effort to fund your own research by seeking out fellowship and grant opportunities. You may also be asked to contribute to larger lab proposals (e.g., by sharing preliminary data or providing edits).
  
\item Mentor undergraduate researchers. This involves both overseeing undergrad helpers in the lab, and serving as co-advisors on senior theses for undergrads who exhibit interest and promise in engaging in the lab more substantively. Mentoring thesis students typically involves:
\begin{enumerate}
\item Working with Jeff and the undergrad to identify a thesis project. Ideally this will involve carving off a subproject that is related to and supports your own work.
\item Providing the undergrad with training on protocols and methods.
\item Meeting with the undergrad to discuss the literature and ensure they understand the ``why'' as well as the ``what'' of what they are doing.
\item Directing the undergrad to funding opportunities (e.g., UROP, SPUR, O'Day) and providing feedback on their proposals.
\item Workshoping thesis drafts and presentations with the student.
\item Providing guidance for students on next steps after graduation (e.g., providing feedback on grad school inquiries/applications for inclined students).
\item Keeping track of highlight points and examples for letters of recommendation, and working with Jeff to write these.
\item Communicating with Jeff about the process and seeking his guidance if undergrads are not maintaining their end of the agreement.
\end{enumerate}

\item Contribute to assigned lab tasks (e.g. keeping the lab tidy, organizing lab meetings, etc.)

\item Attend research conferences, especially in the last few years of your program. Lab funds for travel are limited for non-grant related work, so you must try to  fund yourself through grants from the conference and department.
  
\item Be generally present during the academic school year, except for fieldwork, conference travel, etc. Jeff will not keep track of when you are in the lab or office, but you are expected to be present enough for the casual interactions and discussions that fuel collaboration with your labmates and colleagues.  

Grad school is a long time, and life happens. Jeff will work with you to find a balance between personal needs and professional progress. If you anticipate a reason to be away from the lab for an extended time, communicate with Jeff to work out a plan together. Note that flexibility here will depend on a number of factors, including the need, the available funding, and your ability to engage remotely.
  
\item Follow lab protocols such as equipment checkout, data and project management, lab safety, etc.

\end{enumerate}


\section{Undergraduates - What Jeff expects of you}
\begin{enumerate}
\item Be an engaged member of the lab group and contribute to an inclusive and collegial environment. This includes:
\begin{enumerate}
\item Attending lab events, such as lab meetings, when possible, and communicating when you have conflicts (e.g., we try to schedule events to avoid class time, but it's not generally possible to accommodate everyone)
\item Paying attention and taking good notes when learning protocols and methods, and paying it forward in terms of teaching others 
\item Engaging with others in the lab and providing feedback on each other's experimental design, proposals, talks, and drafts
\item Treating one another with courtesy and respect
\end{enumerate}

\item All students are expected to commit to a schedule and be willing to contribute beyond the schedule in emergencies, if possible.
Please be proactive and communicative about deadlines and scheduling conflicts as they arise.

\item All students are expected to understand the ``why'' as well as the ``what'' of what they are doing - this involves actively asking questions when you are curious or confused, reading the relevant scientific literature, seeking out additional relevant literature and discussing what you've read with Jeff and/or graduate students.

\item When new to the lab, undergraduate work will initially involve helping on projects led by graduate students and/or Jeff. We generally anticipate at least two quarters of work in a supporting role that demonstrates commitment, enthusiasm and attention to detail, prior to developing independent projects. 

\item After approx. two quarters you're highly encouraged to conduct independent research, which can be via a thesis or otherwise. At this stage, we expect you to build your own research ideas in collaboration with a graduate student mentor in the lab and Jeff. This involves spending time searching and reading the literature, deep thinking and brainstorming, and being receptive and responsive to feedback from your mentor and the lab group.

\item Students interested in conducting a thesis:
\begin{enumerate}
\item are expected to make an effort to fund your own research by seeking out fellowship and grant opportunities, responding when Jeff or your grad/postdoc mentor directs you to funding opportunities, and putting care and attention into proposals. Common funding opportunities include UROP, SPUR, UnderGrEBES, and the O'Day fellowship. 

\item are expected to enroll for formal thesis requirements in their program (most commonly Biology, ESCI, ENVS, or the Honors College), and to stay on top of relevant deadlines.

\end{enumerate}
\end{enumerate}

\end{document}



%%% Local Variables:
%%% mode: latex
%%% TeX-PDF-mode: t
%%% End:


