\documentclass[12pt]{article}
\usepackage[top=0.8in, bottom=0.8in, right=0.8in, left=0.8in,
paperwidth=8.5in, paperheight=11in, nohead]{geometry}
\geometry{letterpaper}
\usepackage[pdftex]{graphicx}
\usepackage{color}
\usepackage[normalem]{ulem}
\usepackage{amssymb}
\usepackage{amsmath}
\usepackage{epstopdf}
\usepackage{setspace}
\usepackage{mdwlist}
\usepackage{hyperref}

\begin{document}

\title{General life/academic advise}
\author{Lauren C. Ponisio}

\maketitle

\section{Set goals}
\label{sec:goals}
Set everything from long-term goals to daily goals. If you do not,
then time will just slip away. Continuously evaluate your progress
toward the goals you have set. If you are falling behind in reaching
those goals, ask yourself why, then do something about it.

\textbf{You are working for yourself} (and more broadly society and
biodiversity, no pressure). The harder and more effectively you work,
the better it is for you. Do not, however, just ``put in hours''. Work
hard and concentrate hard, and enjoy the work and concentration.

Strategies I find effective for time management (from Deep Work, by
Cal Newport, Getting things Done by David Allen)
\begin{enumerate}
\item Keep track of the number of hours of focused, deep work you do
  each day. Strive for four hours. Deep work includes something you
  cannot train someone to do in a few days. Examples include writing,
  coding, reading a manuscript, identifying specimens. Shallow work
  includes pinning/sorting/labeling specimens, filling out forms,
  meetings, most emails. Don't let your percent of deep work fall
  below your shallow work. Have a bad day where you did not get much
  deep work done? Work out why and avoid this in the future.
\item break tasks into actions, and have a sensible task manager (like
  Omnifocus?). Transfer all of the tasks in your brain into that task
  manager (including things like buy batteries, etc.).
\end{enumerate}

Each year will we will make a professional development plan, and
discussion our plans and a group, and one-on-one with me.

\section{Begin to imagine your research program}
\label{sec:research}

If you plan to stay in academia, begin to this of what your lab's
theme/research mission will be. Cultivate answers to the following
questions:

\begin{enumerate}
\item So, what do you do?
\item How does your work fit into the ``big picture'' -- what major
  questions does it address?
\item How do you differentiate your work from your Ph.D or
  postdoctoral adviser's work?
\end{enumerate}


\section{Learn to talk science}
\label{sec:talkScience}

From John Thompson, a highly respected evolutionary biologist:

\textit{You will spend much of the rest of your life trying to explain
  concepts, hypotheses, and results to others. The ability to do so
  will not develop miraculously. You must learn from experience how to
  get your point across in research seminars, in classrooms, and in
  meetings with people outside your discipline. If you want to
  convince colleagues that you have something important to say, you
  need to be able to keep them awake and interested during a seminar
  or a discussion. Think about how often you have been bored by having
  to listen to a speaker who wastes an hour of your time as he or she
  mumbles or reads to you --- slide after slide --- a disjointed talk that
  makes no important or interesting point. The same applies to giving
  lectures to students. With so many capable scientists competing for
  jobs, universities should be able to keep only those faculty who are
  both good researchers and good teachers. With the keen competition
  for jobs that now occurs, that is what will happen more often in the
  future.}

\textit{So get all the experience you can get and learn from your
  mistakes. Watch carefully how others give seminars and
  lectures. Take the best from what you see in them and work out which
  of those techniques will work well for you. The structure of a good
  talk is completely different from the structure of a scientific
  paper. Your goal should be not only to convey information on your
  recent work but also to put that information into the kind of
  broader context that is not possible in a scientific paper. The most
  boring talks are those are nothing more than a description of the
  methods and an endless series of tables and graphs. Your audience
  deserves more than these details, as important as they are. The
  audience deserves to hear from you what these results mean in a
  broader sense and why they should care.}

I will also add that finding your voice can be difficult. You will
look at your colleagues confidently talk about their projects and
wonder how they are so amazing. They are no more brilliant than you,
they just play the part better. Find a way to instill confidence
within yourself. Celebrate your successes! Stay away from people who
are not your advocates. Don't compare yourself to others, just work as
hard as you can toward your goals. Become comfortable with what you
know, and what you do not know. Treat people like colleagues and you
will be treated like a colleague. Never say the words ``just'' or
``only'' when introducing yourself or describing your work.

\section{Begin to develop your mentoring, outreach and teaching
  philosophies and skills}
\label{sec:skills}

As a scientist my goals are to promote biodiversity conservation
(though research, teaching and outreach) and promote diversity in the
sciences. Maybe you have similar goals?

As a graduate student/post-doc, you will continuously need to make
decisions about what projects and outreach events to focus on. To help
you prioritize, begin to assemble your general goals as a scientist
and choose based on advancing those goals.

Mentoring is at the core of being a scientist. And doing it
effectively is hard. As a lab we will draft mentoring philosophies and
plans for mentoring undergraduates (and graduate students, for
post-docs). Each year we will revisit these plans and adapt them based
on our experiences in the last year.

Some general guidelines I abide by:
\begin{enumerate}
\item Give positive feedback.
\item Find ways to convey that you believe your students are capable,
  and that you believe they have the potential to build on their
  knowledge and skills.
\item Be a whole person, be a witness, but avoid becoming entangled in
  peoples' lives
\item Set expectations of all kinds (for labwork, for the project, for
  the conversation)
\item Be accessible, but try to schedule the time you spend
  helping/giving advice to encourage independence
\end{enumerate}

\section{Seek additional mentors}
\label{sec:mentors}
Expose yourself to different ideas, methods, and mentoring
strategies. Mentors can be faculty, post-docs and graduate
students. Talk with successful people in your profession whenever you
can. Rarely will you find someone you cannot learn something from. If
you want to go into academia, begin now to understand what skills you
need beyond research and teaching to be successful in the long term. If
you want to use your skills to become a policy maker, start talking
now with policy makers and engage in some policy-making activities
when given the chance. These are just two examples. The general point
is that you are not ``just a graduate student''. You are a
professional who is in the process of developing and honing a wide
range of skills and perspectives that will allow you to attain your
goals in an ongoing and seamless way.

For graduate students in particular, start thinking about your
committee dream team early and cultivate relationships with those
faculty. You can take their classes, and if none are offered, send
them a email. Ask them about the directions their lab is going, but do
your research and be able to talk to them about their past
work. \textbf{Faculty members are people too!} I will happily talk
about our lab's future research plans for hours and I will be excited
to meet an early career researcher like yourself! I took becoming a
professor for me to stop being intimidated by talking to professors!
You don't need to wait as long as I did!

To avoid adding to the email/logistical load most faculty face,
however, check to see if they have open office hours before
emailing. If not, write, clear, easy to read emails that suggest a
meeting time. Avoid back and forth on scheduling as much as
possible (see process-centric reply in general logistics document).

Professional meetings are also a great time to meet people you have
academic crushes on. Email them ahead of time, ask to have lunch with
time and discuss their future research directions.

For those thinking about a career in academia, this is particularly
important for helping you design your future lab. I once asked a new
faculty member what she learned most from her post-doc. She replied
``I got to see a different way of running a lab and mentoring''. Don't
wait until your post-doc. Start now. Listen to what people say about
their advisers. The good things and the bad things. How would you do it
differently? Also, we adaptivly manage the Ponisio lab, so perhaps we
can try out new strategies in our lab and see what happens.

\subsection{How to get help}
\label{sec:help}
Though we are all in the same boat as far as trying to promote
biodiversity conservation and science, the structure of academia means
that helping people can hurt you (in the sense that you are not
spending time on your own projects). When encountering challenges with
the science, lab work or computational challenges, first sit down and
think about solutions yourself, then look for answers in the
literature, then solicit advice from fellow lab-mates, students, and
post-docs, then seek advice of the PI. Some tips for getting
colleagues to help:
\begin{itemize}
\item make sure to thank people sincerely for their help! Like mentee
  relationships, positive feedback is important to colleagues and
  mentors. 
\item if the person you are asking help of can substantially
  contribute to the project, offer her/him coauthorship. It means
  she/he gets tangible credit for their help. Build a network of
  coauthors what compliment your skills
\item try not to interrupt people in their work days and instead
  schedule a time with them
\item if you want to bounce ideas off a person ask for general advise,
  invite her/him to lunch/dinner to keep it as much as possible
  outside of work hours and more social
\item trade help explicitly (e.g., R help for lab help). You don't
  want to ``keep score'' but sometimes it helps when everyone feels
  like they are benefiting from an interaction
\end{itemize}


\section{Be an advocate, fight unconscious bias}
\label{sec:advocate}
Science is done by humans. Humans are social, power dynamics
exist. Humans are biased. As a latina, female scientist who wears
dresses, one of my primary (and ongoing) struggles has been learning
to respond to and combat biases and empower myself. Some strategies:
 
\subsection{The Buddy system}
\label{sec:buddy}
\begin{enumerate}
\item Designate a Bias Buddy (Strongest when different race, gender,
  religion, etc.)
\item Remind each other before meetings, events
\item Call out interruptions. Ex: ``I want to hear what Sarah has to
  say.''
\item Give credit where credit is due. Ex: ``I think Julio just said
  that 15 minutes ago.”''
\item Give emotional support. Ex: ``She was not an advocate to you in
  there.''
\end{enumerate}

\subsection{Responding to unconscious biases}
\label{sec:responding}
When faced with a micro-aggression (from the Unconsciousness Project):
\begin{enumerate}
\item Start with empathy
\item Don't need pitchforks; maybe just cocktail forks (i.e., a
  casual, un-antagonistic conversation)
\item Use I statements, they cannot logic away your feelings and it
  makes people less defensive
\item Explain why you think it's a problem
\item Make explicit requests. This empowers the offending party with
  an action item

\end{enumerate}


\section{Maintain your emotional well-being}
\label{sec:wellBeing}

In graduate school you develop not only as a researcher, but as a
person. Both types of growth are valuable, and will put you on a path
to becoming an amazing scientist, communicator, mentor and
teacher. All of this growth, however, can be overwhelming.

At the same time, the curse of academia is uncertainty. It is
difficult to find funding for research and fellowships, and hard to
find jobs. You spend a lot of time wondering what projects to work on,
what questions are interesting. After your phd you generally have to
move to a new place, leaving your friend network and lab, and start
over again. The same thing happen after your post-doc when you get the
job of your dreams.

Do what you need to stay healthy. If you are overworked, you will not
be doing good science. Take a break, come back re-charged. Take
vacations. Set a time during the week where you are not allowed to do
work (Saturday perhaps?) and maximize the utility of that day (i.e.,
do something that makes you enjoy life and feel excited about the week
ahead, not just chores).

If you need to talk to me about sometime in your life that is effecting
your personal/academic well-being, please do so. I will be happy to
listen. 

But, I highly recommend that everyone sees a therapist. UC mental
health services are amazing, and copays are \$15/10 graduate/post-doc
for unlimited sessions.

\end{document}

%%% Local Variables:
%%% mode: latex
%%% TeX-PDF-mode: t
%%% End:


